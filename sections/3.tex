\section{U-net的实现要点}

\begin{frame}[allowframebreaks]
  \frametitle{\textsc{目录}} \vspace{-0.3cm}
    \begin{spacing}{0.0}
        \tableofcontents[currentsection,hideallsubsections]
    \end{spacing}   % 若不想要目录, 注释掉该句
\end{frame}



\begin{frame}[fragile]
    \noindent\large\textbf{实现网络模型}

    \vspace{1em}
    U-net的实现网络模型并不复杂,使用到的操作有:卷积,转置卷积、池化、激活函数、批归一化、中心裁剪、拼接
    \begin{lstlisting}
nn.Conv2d(c_i,c_o,3)            #卷积
nn.ConvTranspose2d(c_i,c_o,2,2) #反卷积
nn.functional.max_pool2d(x,2,2) #最大池化
nn.ReLU()
nn.BatchNorm2d(c_o)             #批归一化
x = x[:,:,c:-c,c:-c]            #中心裁剪
torch.cat([x1,x2],dim=1)        #拼接
    \end{lstlisting}
\end{frame}

%\section{应用}


